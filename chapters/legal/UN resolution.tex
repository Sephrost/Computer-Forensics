\chapter{UN Resolution on Cybercrime}

\section{Background and Objectives of the Resolution}
The primary objective of this resolution is to \textbf{initiate} the
\textbf{drafting} of a \textbf{global treaty} aimed at
\textbf{combating cybercrime through multilateral negotiations}. The
title, "Countering the use of information and communications
technologies for criminal purposes," signifies a step towards
developing an international convention on cybercrime, targeting the
implementation of concrete measures against this form of crime.

The resolution establishes a dedicated Committee responsible for
drafting a \textbf{comprehensive convention}. This process is expected
to be \textit{transparent}, involving a wide range of stakeholders,
such as developing countries, intergovernmental organizations, and
field
experts.

\section{Impact on International Cybersecurity Policies and Practices}

\subsection{UN Resolution (26 May 2021)}
The adoption resolution could significantly impact international
cybersecurity policies, promoting greater cooperation and
harmonization among states. By \textbf{establishing shared minimum
standards} through an international cybercrime treaty, it fosters a
more unified regulatory framework and enhances investigative
cooperation across different jurisdictions.

The adoption of the resolution required various amendments and
compromises, highlighting the importance of \textbf{balancing national
and supranational}interests while ensuring inclusivity and
transparency.

\subsection{Human Rights and Privacy Risks}
One of the central issues discussed in the UN report is the need to
balance cybersecurity measures with the protection of human rights,
particularly \textbf{privacy and data protection}. Although there is a
consensus on the importance of cybersecurity, the UN GGE report warns
against overreach, as unregulated measures may infringe on civil
liberties.

\subsection{Emerging Threats and Supply Chain Integrity}
The report emphasizes the growing risks associated with
\textbf{vulnerabilities in global supply chains}, especially in the
ICT sector. Such vulnerabilities could lead to large-scale attacks or
espionage, underlining the need for states to secure digital
infrastructures and collaborate on emerging threat information.

\section{Future Legal Framework for Cybersecurity}
Cybersecurity policymakers are required to implement stringent
measures to \textbf{protect data during investigations}. Legislative
provisions should allow for periodic review and updates to
cybersecurity practices, ensuring adaptability to evolving digital
threats.

\section{Analysis of Key Components and Legal Implications of the
Resolution}

\subsection{Ad Hoc Committee}
The resolution establishes an Ad Hoc Committee to draft a global
cybercrime convention. This committee is mandated to convene at least
six times, each session lasting 10 days, beginning in January 2022.
Decisions within the committee are expected to be made by consensus
or, failing that, by a \textbf{two-thirds majority}.

\subsection{Cyber Sovereignty and Legal Boundaries}
The concept of cyber sovereignty presents a significant legal
challenge. Countries such as China and Russia advocate for state
control over cyberspace, potentially clashing with international norms
of internet freedom and openness. The UN resolution seeks to address
these tensions, but the broader debate over the degree of state
control in cyberspace governance remains unresolved, necessitating a
balance between sovereignty and global cooperation.

\subsection{Public-Private Cooperation and Liability}
Legal frameworks must account for the increasing role of private
companies in cybersecurity. The resolution encourages collaboration
between states and private entities, such as internet service
providers and cybersecurity firms, to counter cyber threats. This
approach raises legal questions concerning the responsibility and
liability of these companies, especially when they participate in
responding to or preventing cyberattacks.

\section{Jurisdictional Provisions (Article 22)}

\subsection{Territorial Jurisdiction}
State Parties must establish jurisdiction over offenses committed
within their territory or on vessels or aircraft registered under
their laws.

\subsection{Extended Jurisdiction}
States may also establish jurisdiction over offenses that:
\begin{itemize}
  \item Are committed against their nationals.
  \item Are committed by their nationals or stateless persons
    habitually residing within their territory.
  \item Are committed outside their territory with the intent of
    carrying out an offense within their territory, as specified in
    Article 17 of the Convention.
  \item Are committed against the State itself.
\end{itemize}

\subsection{Jurisdiction and Non-Extradition}
If the alleged offender is present within a State's territory and not
extradited solely due to nationality, the State must establish
jurisdiction over the offense. In cases where extradition is denied
for other reasons, States may take additional measures to establish
jurisdiction.

\subsection{Coordination Among States}
When a State exercising jurisdiction becomes aware of other States
conducting investigations or proceedings for the same offense,
authorities are encouraged to consult and coordinate their actions.

\subsection{Compatibility with International Law}
This Article affirms that the Convention does not prevent any State
Party from exercising other forms of criminal jurisdiction, as
permitted by its domestic law.

\section{Garlasco Case}
Digital evidence could be altered and can contain countless pieces of
information. The “Garlasco” case is a clear example of this.

Alberto Stasi was acquitted of the murder of his girlfriend, Chiara
Poggi, by the Court of First Instance in December 2009, and the
judgment was confirmed in the Appeal court in December 2011.

\section{Italian Case Law on Digital Forensics}
\begin{itemize}
    \item \textbf{13/08/07:} Stasi wakes up at 9, telephones Chiara
      Poggi, works on his thesis.
    \item \textbf{14/08/07:} Chiara Poggi died between 10:30 and
      12:00.
    \item \textbf{29/08/07:} Stasi voluntarily hands over his PC to
      the Police.
    \item \textbf{17/12/09:} Judge Vitelli acquits Stasi of murder.
\end{itemize}
The expert report requested by the judge shows that Stasi was working
on his thesis during the period when Chiara Poggi was killed.

\section{The Internet of the Human Body: Towards a Habeas Data?}
\begin{quote}
    “If your internet thermostat's pinging servers all day, will the
    cops think you're a weed farm? Or just a hot yoga gym?" -
    \textbf{Jonathan Zittrain}
\end{quote}

\begin{quote}
    “Sure, encrypt your email – while your shiny IoT toothbrush spies
    on you” - \textbf{Susan Landau}
\end{quote}

\section{Cases}

\subsection{Connie Debate Case: Fitbit}

“As people continue to provide more and more personal information
through technology, they have to understand we are obligated to find
the best evidence, and this technology has become a part of that.”
\textit{Detective Christopher Jones - East Lampeter Township Police
Department in Pennsylvania}

For example, in some legal relevant cases, the location tracking could
be used to understand if a person is lying or not. And another
important aspect is timing, because the seizure of the device must
happen as soon as possible, of course.

“We are entering an era of sensorveillance. People are just waking up
to the fact that their smart devices are going to snitch on them and
that they are going to reveal intimate details about their lives they
did not intend law enforcement to have”  \textit{Andrew Ferguson, a
University of the District of Columbia law professor}

\subsection{James Bate Case: Amazon Echo}
Another similar one is this one, in which evidence could be acquired
trough the microphone of an Amazon Echo. In this case there's not a
jurisdiction issue(the crime was committed in the US), but the
willingness of handing out the data and the Terms of Service(TOS),
because companies can create some Dark Patterns to make the user 
do something that they don't want to do, for example accepting the
TOS.

“The Amazon Echo device is constantly listening for the 'wake' command
of 'Alexa' or 'Amazon,' and records any command, inquiry, or verbal
gesture given after that point”  \textit{Search Warrant}

The Amazon TOS prohibits the disclosure of information to a third
party (except Amazon or the user) without the consent of Amazon
itself. In this case the user requested the disclosure to the law
enforcement, which amazon agreed to.

“The allegation that the Echo is possibly recording at all times
without the wake word being issued is incorrect”  
\textit{Answer of an Amazon Representative}

To sum this up, jurisdiction is not the only problem, but also the TOS
between the user and the service provider, especially in case of IoT
devices, which are mostly produced in China.

\subsection{Ross Compton Case: Pacemaker}

This is the last mentioned IoT case. In this case, the Ross Compton
house was on fire and he claimed that he was able to escape the house 
and throw some of his belongings out of the window. In this situation
the police wanted to know the beat rate of his pacemaker, to
understand if it was compatible with that kind of situation.

“There is a lot of other information about things that may
characterize the inside of my body that I would much prefer to keep
private rather than how my heart is beating. It is just not that big
of a deal”  \textit{Judge Charles Pater}

“Americans shouldn't have to make a choice between health and privacy.
Compelling citizens to turn over protected health data to law
enforcement erodes those rights.”  \textit{Electronic Frontier
Foundation Attorney Stephanie Lacambra}

When dealing with this kind of sensitive informations, a choise has to
be made if the erosion of privacy is worth the gain in security.
In this case, the evidence was accepted by the court.
\section{Facial Recognition Biometric Border}
This kind of measure are still be discussed, and will be, because they
are a very sensitive topic.

“U.S. Customs and Border Protection says it will delete the live
photos captured at the gate within 14 days for citizens, and that it
only uses them to verify identity by comparing them with the database
photos”  \textit{CBP Privacy Impact Assessment}

“Face Recognition has a great potential for expansion and misuse: for
example, you can subject thousands of people to face recognition when
they’re walking down the sidewalk without their knowledge”
\textit{Senior Policy Analyst, ACLU - Jay Stanley}

\section{Categories of Law Enforcement Activity}
When a crime has to be "discovered", there are different scenarios
that may occur:

\begin{itemize}
    \item Situations involving officers observing an ongoing crime
    \item Situations involving officers investigating a past crime
    \item Situations involving officers predicting a future crime
\end{itemize}

In those contexts, the use of technology is not so different, for
example in the case of past crimes, the \textbf{Keycrime} software can
be used to analyze possible suspects based on the heuristics that
criminals specialize in certain crime types. This has been sentenced
as a "predictive" approach(which is inaccurate according to professor
Vaciago) and forbidden in the EU according to the AI act. On the other
hand, there's a possibility of predicting future crimes with tools
such as \textbf{PredPol}, which is a software that uses data to
predict where crimes are likely to occur. However accurate they may
be(around subway stations, they registered a crime reduction rate of
30\%), they are still forbidden in the EU.

\begin{boxH}
  Predictive policies are forbidden in the EU, and they most likely
  will be for the foreseeable future.
\end{boxH}



\section{5 Pillars of “Police” Directive}
All the significant opportunities that technologies offer to law 
enforcement authorities must be balanced with the protection of 
fundamental rights. The “Police” Directive is based on five pillars:

\begin{itemize}
    \item Fairly, lawful, and adequate data processing during criminal
      investigations or to prevent a crime
    \item Clear distinction of various categories of data subjects in
      a criminal proceeding (investigated person, person convicted,
      victim of crime, third parties to the criminal offense)
    \item Prohibit measures that produce adverse legal effects for the
      data subject based solely on automated processing of personal
      data
    \item Implementation of privacy by design and by default
      mechanisms to ensure protection of data subject rights and
      minimal processing
    \item Cooperation with relevant supervisory authorities, providing
      all necessary information for their duties
\end{itemize}

\section{Automated Processing (Article 11 - “Police Directive”)}

All automated process and tools require a human supervision to be
used. Not a formal check but a substantial one, to ensure that the
tools are correct and used correctly.

Automated processing is forbidden unless:
\begin{itemize}
    \item There is human intervention.
    \item It produces an adverse legal effect concerning the data subject.
    \item It is authorized by the EU or Member States.
    \item It includes appropriate safeguards for data subject rights
      and freedoms.
\end{itemize}

Profiling that results in discrimination based on special categories
of personal data as per Article 10 shall be prohibited under Union
law.

\section{Transparency, Retention, and Enforcement}

Tools based on big data for law enforcement purposes must be checked
by law enforcement authorities prior to final purchase and verified
for suitability, correctness, and security, with limitations due to
proprietary software.

While the EU legal framework on data retention is still under
development, safeguards are required for lawful data retention per ECJ
case law:
\begin{itemize}
    \item The crime must be serious.
    \item Retention measures must be necessary and proportionate.
    \item National authorities’ access must meet specific data
      protection safeguards.
\end{itemize}

\section{Privacy vs Security}

\textit{What happens if security prevails over privacy?} - Netflix
case

\section{Budapest Convention on Cybercrime - Overview}

The Budapest Convention on Cybercrime was issued by the Council of
Europe on November 23, 2001.  
Italy ratified the Convention with Law n. 48 on March 18, 2008, and it
was published in the Official Gazette on April 4, 2008.
