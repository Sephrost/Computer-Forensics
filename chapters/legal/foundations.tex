\chapter{Foundations of digital forensics}
First of all, lets define what can be considered digital evidence.
There are many definition but the one that is most commonly used,and
the preferred one at the exam, is:
\begin{boxH}
  \textbf{Digital evidence} is \textit{any} information of evidential
  value whether memorized or sent in a \textbf{digital format}.
\end{boxH}
Digital evidence has some defining characteristics: it is invisible to
the untrained eye (it usually not simple to find), it may need to be
interpreted by a specialist, it may be altered or destroyed trough
normal use and it can be copied without limits (you must always work
on a copy to avoid tampering evidence).

Digital evidence, to be admissible in court, must have some
properties:
\begin{itemize}
  \item \textbf{Authenticity}: avoid any digital evidence
    \textbf{tampering}. Only certified formats should be used.
  \item \textbf{Reliable and believable}: the evidence must be redly
    \textbf{understandable} to the judge. Sometimes a report is not
    available to explain the evidence, so it must be self-explanatory.
  \item \textbf{Proportional}: \textbf{respect} fundamental
    \textbf{rights} of parties affected by the measure
  \item \textbf{Admissible}: \textbf{compliant with law} and best
    practices(admissible in court). One thing is to have evidence, the
    other is to have it admissible in court, and that's not always the
    case.
\end{itemize}

There are three main kinds of digital evidence:
\begin{itemize}
  \item \textbf{Created by man}: any piece of digital data that is the
    result of a step or action taken by a human person. Of those,
    there are two types:
    \begin{itemize}
      \item \textbf{Human to human}, such as an email
      \item \textbf{Human to machine}, such as a file saved on a disk
    \end{itemize}
  \item \textbf{Created independently by the computer}: any piece of
    digital data that is the result of the processing of data carried
    out by a software in accordance with a specific algorithm and
    without human intervention (e.g. telephone records or Internet
    Service Provider logs)
  \item \textbf{Created by both man and the computer}: an electronic
    spreadsheet where data is entered by the human, while the computer
    works out the result.
\end{itemize}

Lets now define what digital forensics is:
\begin{boxH}
  \textbf{Digital forensics} is the process of getting hold of
  evidence without modifying the IT system in which the evidence is
  found, ensure that the evidence acquired in another medium is
  identical to the original and analyze the evidence without modifying 
  it.
\end{boxH}

\subsubsection{The "Big Five" of digital forensics}
There are some principle that are to be followed during digital
forensics:
\begin{itemize}
  \item \textbf{Data Integrity}: No action taken should change
    electronic devices or media, which may subsequently be relied upon
    in court. 
  \item \textbf{Chain of Custody}: An audit trail of all actions taken
    when handling electronic evidence should be created and preserved. 
  \item \textbf{Specialist Support} If investigations involving search
    and seizure of electronic evidence it may be necessary to consult
    external specialists. 
  \item \textbf{Appropriate Training}: First responders must be
    appropriately trained to be able to search for and seize
    electronic evidence if no experts are available at the scene. 
  \item \textbf{Legality}: The person and agency in charge of the case
    are responsible for ensuring that the law and the above listed
    principles are adhered to. 
\end{itemize}

\section{Digital Investigation Procedure}
Digital investigation is carried out in many steps.
\subsection{Identify the Suspect}
The first phase is also the most difficult one, because there are many
tools to be anonymous in the internet.
The general approach to this phase is the following:
\begin{enumerate}
  \item An investigator receives a \textbf{complaint} by a victim of
    \textbf{cybercrime} or detect an illegal content on line. (OSINT
    can be used to detect illegal content)
  \item The investigator uses the Court System to compel the ISP to
    \textbf{reveal} a \textbf{physical location} that corresponds
    likely to the source of Network (\textbf{IP Address})
  \item Under a \textbf{warrant} (depending from the Jurisdiction) the location
    is searched and any computer or other device is seized
\end{enumerate}

But why the ISP has to cooperate and give away the IP address?
In the EU it is because \textbf{Data Retention Directive} 2006/24/EC,
that requires that all the data of the communication must be stored
for a certain amount of time, from 6 to 24 months. This directive was
unconstitutional in many countries,
\begin{boxH}
  \textbf{Data retention} (or data preservation) generally refers to
  the \textbf{storage} of call detail records (CDRs) of telephony and
  internet traffic and transaction data (IPDRs) by governments and
  commercial organizations
\end{boxH}
In any case data retention is usually a problem because there's no a
homogeneous law for data retention, mainly for privacy reasons.

Even if there's no data retention law, the request of user data
disclosure have been a lot in the past years have steadily increased,
because the processes have been automated, as you can also see from
figure \ref{fig:disclosure-req}.

\begin{figure}[H]
  \centering
  \includegraphics[scale=.4]{img/disclosure requests.png}
  \caption{Requests for user data disclosure during the years}
  \label{fig:disclosure-req}
\end{figure}
Another instrument that could be used to identify the suspect is the
\textbf{facial recognition}, that can be used only for terrorism
situations, but not in a systematic way.
\subsection{Detecting and Seizing Digital Evidence}
Anyone wanting to seize and validate digital/electronic evidences
(content of an e-mail or an entire hard-disk) has to respect two
fundamental \textbf{rules}: Bit-Stream Copy and Hash Function.
\subsubsection{Bit-Stream Copy}
\begin{boxH}
  A \textbf{bit-stream copy} can \textbf{clone} the \textbf{entire
  drive}.
\end{boxH}
It is a particular form of duplication in which the content of the
physical unit is read sequentially loading the minimum quantity of
data that can from time to time be directed, then recording it in the
same sequence on a standard binary file, generating a physical image
of the original medium.\\
Depending on how \textit{powerful} is the device operating the copy it
can take up to some days.

\subsubsection{Hash Functions}
During the forensic analysis of modifiable media, the Hash 
guarantees the intangible nature of the data that it contains.
\begin{boxH}
  The Hash is a \textbf{one-way function}, by means of which a
  document of random length is converted into a limited and fixed
  length string.
\end{boxH}
This string represents a sort of ‘digital fingerprint’ of the non-
encrypted text, and is called the Hash Value or the Message Digest. If
the document is modified, even to the slightest extent, then the
fingerprint changes as well. In other words, by calculating and
recording the fingerprint, and then recalculating it, it can be shown
beyond all doubt whether the contents of the file, or the medium, have
been altered, even accidentally.\\

In any case we have two main problems while acquiring data: encryption
and jurisdiction. After all, the ISP, TELECO or even a bank does have
to cooperate and give out the data. Another big issue it the cloud
computing aspect, because the location of data is another big problem,
because it can be either:
\begin{itemize}
  \item \textbf{at rest}: does not reside on the device. 
  \item \textbf{in transit}: cannot be easily analysed because of
    encryption. 
  \item \textbf{in execution}: will be present only in the cloud
    instance
\end{itemize}
\textbf{Validation} of digital evidence is a very important step,
because it is the only way to prove that the evidence is authentic.
This is especially important for proof found on the internet. There
are some tools that can be used to validate digital evidence, such as
Web Forensics.\\
Another issue that has to be accounted for is the \textbf{chain of
custody}, that is the process of maintaining and documenting the
location and handling of evidence. After all, the bit is eternal, but
the storage medium is not.
\subsection{Analysis of Digital Evidence}
\begin{enumerate}
  \item \textbf{Text searches}: aimed at scanning files, directories
    and even entire file systems for specific text terms. Nowadays,
    this is carried out with AI(only for committed crimes, to avoid
    social control) or by using natural language processing tools.
  \item \textbf{Image searches}: aimed at identifying image files in
    various formats, and at generating still frames of digitally
    stored video footage
  \item \textbf{Data recovery}: aimed at recovering all files stored
    on mass memory units, including deleted or damaged data, even if
    a backup is available(just the intention is enough to be 
    considered a crime)
  \item \textbf{Data discovery}: targeted at accessing hidden,
    encrypted or otherwise protected data
  \item \textbf{Data carving}: focused on reconstructing damaged files
    by retrieving portions of their content
  \item \textbf{Metadata recovery and identification}: this digital
    forensic tool is particularly useful for retracing the timeline of
    web accesses and file changes
\end{enumerate}
All those activities should be reported to respect the chain of 
custody.\\ 
Digital forensics analysis can be either \textbf{repeatable} or
\textbf{non-repeatable}. The first case is when we are able to do a
bit stream copy of the data, eg: the correct procedure, because we are
able to do the analysis on the various copies that can be made.\\
In some cases we are unable to do a bit-stream copy, eg: encryption or
jurisdiction issues as in case of cloud computing, the analysis is 
non-repeatable, which some relevant implication on the legal
standpoint. For example the evidence can be collected, in case of live
forensics activity, with the presence of the defendant and his
attorney to make them admissible in court.\\
Another big discussion is the difference between open source or closed
source tools, because every doubt on the tools can be solved by the
transparency of the code.\\
\subsection{Presentation in Court}
The last phase of the digital investigation is the presentation of the
evidence in court, and it's the most important step.\\
This stage is of key importance for Prosecutors, Judges and lawyers,
as the outcome of the trial will depend not only on results achieved,
but also the degree of \textbf{clarity} and \textbf{comprehension} of
the report.\\
Some operations that are recommended to be done are:
\begin{itemize}
  \item the presence of an index
  \item the presence of a glossary and reference notes if there are
    any technical terms
  \item a timeline table and flow charts
  \item some presentation slides with photos
  \item a possible video-recording of operations carried out
\end{itemize}

\section{Encryption during digital forensics}
Encryption is a big problem in digital forensics, and that's pretty
obvious.\\ 
To explain the issue, a good example is the United States v. Boucher
case, Sebastien Boucher's laptop computer was inspected when he
crossed the border from Canada into the USA at Derby Line, Vermont.
Law Enforcement seized the laptop, questioned Boucher and then
arrested him on a complaint charging him with transportation of child
pornography in violation of 18 U.S.C. 2252A\\
When the laptop was switched on and booted, it was not possible to 
access its entire storage capability. This was because the laptop had been protected by 
PGP Disk encryption.\\
A grand jury subpoenaed the defendant to provide the password to 
the encryption key protecting the data, but this is a violation of the
5th emandment, because the password is a self-incriminating
evidence.\\
\begin{boxH}
  \textbf{Mandatory Key Disclosure} is the legislation that requires
  individuals to surrender cryptographic keys to law enforcement.
  Nations vary widely in the specifics of how they implement key
  disclosure laws.
\end{boxH}
Many countries tried to enforce it, but many don't, like Italy.\\
To this point there are many issues. For a technical point of view, an
expert could always find a way to hide a file, but its also a possible
violation of European Convention on Human Rights: Article 6 Everyone
charged with a criminal offence shall be presumed innocent until
proved guilty according to law.\\
A possible solution is remote forensics, that is the possibility to 
access the data remotely, but has been determined unconstitutional.
\section{Jurisdictional issues}
The “loss of location” of digital evidence in the cloud world creates
problem of jurisdiction. With cloud computing, are the documents
governed by the law of the state in which they are physically located
or by the location of the company possessing them or by the laws of
the state where a person resides? Over the last few years, various
approaches have been offered to solve this problem.

\begin{figure}[H]
  \centering
  \includegraphics[scale=.6]{img/jurisdiction problem.png}
  \caption{Jurisdictional issues}
  \label{fig:jurisdiction}
\end{figure}
We have 4 possible principles to solve the “loss of location” in a 
cloudy world:
\begin{itemize}
  \item Territorial principle: the Court in the place where the data
    is located has jurisdiction.
  \item Nationality principle: the nationality of the perpetrator is
    the factor used to establish criminal jurisdiction. This is a
    really discriminatory principle.
  \item “Flag principle”: which basically states that crimes committed
    on ships, aircrafts and spacecrafts are subject to the
    jurisdiction of the flag state.
  \item “Power of Disposal Approach”: from a practical point of view,
    a regulation based on the power of disposal approach would make it
    feasible for law enforcement to access a suspect’s data within the
    cloud.
\end{itemize}
