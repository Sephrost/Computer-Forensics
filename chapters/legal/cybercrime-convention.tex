\chapter{Cybercrime Convention}
Because this part it \textit{really boring} we will start with an
example.
\section{E-commerce on the Dark Web}
The Dark Web provides a \textbf{platform} for buyers and sellers to
engage in e-commerce transactions, often involving \textbf{illicit
goods and services}, like guns, fake documents etc.. This anonymous
marketplace operates with the same principles as traditional
e-commerce, but with heightened security and privacy measures to
conceal identities.\\
Vendors on the Dark Web offer a wide range of products, from drugs and
weapons to stolen data and hacking services. Buyers can browse
listings, read reviews, and complete purchases using cryptocurrencies,
all while maintaining a high degree of anonymity.
\subsection{Silk Road}
Silk Road, often referred to as the "eBay of drugs," was an
\textbf{online marketplace}that facilitated the sale of a wide range
of illegal substances, including narcotics and controlled substances.
At its peak in 2013, Silk Road had a reported annual revenue of \$89.7
million.
\begin{itemize}
  \item \textbf{Combining Tor, PGP, and Bitcoin}: Ross Ulbricht
    leveraged the anonymity of Tor, the encryption of PGP, and the
    decentralized nature of Bitcoin to create the Silk Road marketplace. 
  \item \textbf{Bitcoin-only Payments}: Silk Road required all
    transactions to be conducted using Bitcoin, providing an added layer
    of anonymity and making it harder to trace purchases.
  \item \textbf{User-friendly Interface}: Silk Road featured a well-designed
    interface that allowed users to easily navigate the site and leave
    feedback on their transactions.
  \item \textbf{Intermediary Role}: Silk Road acted as an intermediary
    between buyers and sellers, handling the payment process and
    logistics of shipping items purchased on the marketplace. This is
    one of the main sources of success of the platform.
\end{itemize}
At its peak, Silk Road had over 950,000 registered accounts, 1.2
million transactions, and nearly \$79 million in commissions. Ulbricht
was arrested in 2013 due to some critical mistakes, like using its
personal email to advertise the website and having counterfeit
documents delivered to his home address when he found out that he was
on investigation.\\
Ulbricht faced 7 key charges, including drug trafficking,
money laundering, and computer hacking, which were consolidated into 3
main counts against him and the trial ended in just 13 days with a
life sentence plus 180\$ million in damages.\\
What we can learn about this? First, the anonymity is really an issue
with ongoing investigations, but everyone can make mistakes, which
means that eventually, the law will catch up with you. Second,
everyone can create this kind of platform, which implies that is
necessary to solve the issue. For this purpose, cooperation is most
fundamental, and the Budapest Convention is the first step in this 
direction, by creating standards and protocols for international
cooperation in this field.

\section{Budapest Convention}
The Budapest Convention on Cybercrime is an international
\textbf{treaty} that aims to address the challenges posed by
cybercrime and promote cooperation among countries in combating these
offenses.
It has three main purposes:
\begin{itemize}
  \item Harmonizing national laws on cybercrime
  \item Improving investigative techniques
  \item Increasing international cooperation
\end{itemize}
In involves \textbf{68 countries}, including the United States,
Canada, and many European nations.
\begin{boxH}
  Keep in mind that the council of Europe and the European union are
  two different entities. The first one is able to invite entities
  that are outside of the latter one.
\end{boxH}

Some countries opposed to the convention like India, which has since
reconsidered, and Russia, which rejects it due to concerns about
sovereignty and refuses to cooperate too.\\
Other states has signed but no ratified (Ireland and South Africa),
which means that they are not bound by the convention(in this case is
just a principle declaration, without legal value).
\begin{boxH}
  The \textbf{difference} between signing and ratifying a treaty is
  that \textbf{signing} a treaty is a \textbf{preliminary step} that
  indicates a country's \textbf{intention to be bound} by the treaty,
  while \textbf{ratifying} a treaty is the \textbf{formal act} of
  \textbf{accepting} the treaty and agreeing to be bound by its terms.
  A country can sign a treaty without ratifying it, but it cannot
  ratify a treaty without signing it first.
\end{boxH}
The convention aims to help in the \textbf{fight against crimes} that
can only be committed through the \textbf{use of technology}, where
the devices are both the tool for committing the crime and the target
of the crime, and crimes where technology has been used to enhance
another crime, such as fraud. It \textbf{provides guidelines} for any
country \textbf{developing domestic laws on cybercrime}and serves as a
basis for international cooperation between parties to the convention
but was later expanded to cover other areas of crime, such as hate
crimes.
\subsection{Original key provisions}
The original convention was adopted in 2004 and covers:
\begin{itemize}
  \item the \textbf{criminalisation} of \textbf{conduct}, ranging from
    illegal access, data and systems interference to computer-related
    fraud and dissemination of child abuse material;
  \item \textbf{procedural powers} to \textbf{investigate cybercrime}
    and secure electronic evidence in relation to any crime;
  \item \textbf{efficient international cooperation} between parties.
\end{itemize}

Parties are members of the Cybercrime Convention Committee and:
\begin{itemize}
  \item \textbf{share information} and \textbf{experience};
  \item \textbf{assess implementation} of the convention;
  \item \textbf{interpret} the convention through \textbf{guidance
    notes}.
\end{itemize}

Of the 27 Member States, 26 have ratified the convention. Ireland
has signed but not yet ratified it.

\subsection{Additional Protocol 1(2006)}
This protocol extends the scope of the convention to cover xenophobic
and racist propaganda disseminated through computer systems, providing
more protection for victims.\\
It furthermore:
\begin{itemize}
  \item reinforces the legal framework through a set of guidelines for
    criminalising xenophobia and racist propaganda in cyberspace;
  \item enhances the ways and means for international cooperation in the
    investigation and prosecution of racist and xenophobic crimes online.
\end{itemize}

\subsection{Additional Protocol 2(2024)}
This protocol aims to further enhance international cooperation.\\
It addresses the particular challenge of \textbf{electronic evidence}
relating to cybercrime and other offences being held by service
providers in foreign jurisdictions, but with law enforcement powers
limited to national boundaries.\\
Its main features are:
\begin{itemize}
  \item a new \textbf{legal basis} permitting a \textbf{direct request
    to registrars} in other jurisdictions to obtain domain name
    registration information;
  \item a new \textbf{legal base} permitting \textbf{direct orders to
    service providers} in \textbf{other jurisdictions} to obtain
    subscriber information;
  \item \textbf{enhanced} means for \textbf{obtaining subscriber
    information} and \textbf{traffic data} through
    government-to-government cooperation;
  \item expedited \textbf{cooperation} in \textbf{emergency
    situations} including the use of joint investigation teams and
    joint investigations.
\end{itemize}

\subsection{New Global Cybercrime Treaty}
The Budapest Convention has been a model for other countries and 
regions in developing their own cybercrime laws and treaties.\\ 
The \textbf{United Nations} has also been working on a new global 
cybercrime treaty, which would build on the Budapest Convention and 
expand its scope to cover new forms of cybercrime and emerging 
technologies.\\
We will see it later on.

\section{Harmonization of national laws and international cooperation}
\subsection{International Cooperation Provisions}
This provision s based on a very simple principle.
\begin{boxH}
  Parties are to cooperate "to the \textbf{widest extent possible}" in
  investigating electronic evidence.
\end{boxH}
This is necessary because mutual assistance is often slow and can even
take months to complete(3 up to 6).  The Mutual assistance agreement(
thanks to the Cybercrime convention) allows for expedited requests
using "expedited means of communication", which means that, to an
expedited request, one must provide adequate levels of security and
authentication(eg. encrypted communication).\\
Furthermore, parties may share information without a formal request if
it would assist in investigations or help the receiving party with any
related offences.
\subsection{Mutual Assistance Provisions}
Procedural Powers for Assistance grant the ability to expedite the
preservation of stored computer data, as well as the disclosure of
traffic data. In addition, they allow for the real-time collection of
traffic data and the interception of content data. Assistance in these
matters is provided in accordance with domestic laws and applicable
treaties, subject to any reservations.\\
Article 23, which outlines the General Cooperation Principle,
emphasizes that mutual assistance should be offered to the widest
extent possible. This applies specifically to cyber-related offenses
and the collection of electronic evidence for any criminal offense.\\
However, there are restrictions to this cooperation. These limitations
may arise in cases related to extradition, mutual assistance in the
real-time collection of traffic data, and the interception of content
data.
\subsection{24/7 Network for Immediate Assistance}
Each party is required to designate a contact point that is available
24/7. The primary purpose of this contact point is to provide
immediate assistance in cybercrime investigations, legal proceedings,
or the collection of electronic evidence. This system is modeled after
the G8 network of contact points.\\
The goal of this provision is to expedite the processing of urgent
mutual assistance requests by overcoming delays associated with
traditional bureaucratic channels.

\section{Legal measures against computer-related fraud and forgery}
One of the main legal measures against computer-related fraud and 
forgery is the \textit{criminalization of Fraud and Computer-related
Forgery}.\\
The Convention mandates State Parties to criminalize specific
conducts, including fraud and forgery carried out through computer
systems. This includes digital document forgery and fraud involving
electronic data used for deceit or financial gain. 

\begin{boxH}
  Computer-related fraud involves using computers to gain economic
  benefits through deceit, while forgery includes altering or creating
  digital documents with the intent to mislead.
\end{boxH}

To effectively address these crimes, the Convention introduces
procedural law tools that allow for quicker and more effective
investigations.

For example, expedited preservation of volatile data and seizure of
information are crucial tools for gathering evidence in investigations
against digital fraud and forgery.

The Convention mandates that criminal justice authorities must be
able to use effective means, such as search and seizure, and access
stored data in computer systems, regardless of the type of crime
involved.
\subsection{Computer-Related Forgery}
Each Party shall adopt such legislative and other measures as may be
necessary to establish as criminal offences under its domestic law,
when committed intentionally and without right, the input,
\textbf{alteration, deletion, or suppression of computer data},
resulting in inauthentic data with the intent that it be considered or
acted upon for legal purposes as if it were authentic, regardless
whether or not the data is directly readable and intelligible. A Party
may require an intent to defraud, or similar dishonest intent, before
criminal liability attaches, this is to avoid criminalizing 
accidental behaviour because data alteration is so easy to do.
\subsection{Computer-Related Fraud}
Each Party shall adopt such legislative and other measures as may be
necessary to establish as criminal offences under its domestic law,
when committed intentionally and without right, the causing of a loss
of property to another person by:
\begin{enumerate}
  \item any input, alteration, deletion or suppression of computer data,
  \item any interference with the functioning of a computer system 
\end{enumerate}
with fraudulent or dishonest intent of procuring, without right, an
economic benefit for oneself or for another person.

\section{Procedural powers for law enforcement}
In the past years, we moved from a synopticon world (where the 
many watch the few) to a omnipticon world (where the many watch the
many). Think to a youtube video as an example. In this kind of world,
information control is much more difficult.

The main concern arising for private citizens, companies and public
administration using cloud technologies is not so much the possible
increase in "cyber" fraud or crime than the loss of control over one's
data.\\
This concern is not only for privacy reason, but for digital
investigation purposes.\\

All this is to say that we mainly lost control of our data, and this
is a big jurisdictional problem, because this is usually in the hands
of private companies, following their terms and conditions.
\subsection{Scope of Procedural Provisions (Article 14)}
Each Party must adopt legislative measures to define the powers and
procedures for specific criminal investigations or proceedings.

The provisions apply to offenses covered by the Convention, all other
offenses committed through computer systems, and all electronic
evidence related to any crime.
\subsection{Conditions and Safeguards (Article 15)}
The application of powers and procedures must ensure adequate
protection of human rights, following national law and international
conventions (e.g., European Convention on Human Rights).

Conditions and safeguards include judicial or independent supervision,
considering proportionality and the rights of third parties.
\subsection{Expedited Preservation of Stored Data (Article 16)}
Authorities must be able to order or obtain the rapid preservation
of specific computer data, particularly if there is reason to believe
that the data is vulnerable to deletion or modification.

This order may require the data's custodian to preserve the data for
up to 90 days, extendable as needed.

\subsection{Expedited Preservation and Disclosure of Traffic Data
(Article 17)}
To ensure data preservation, authorities can demand rapid disclosure
of traffic data to identify service providers and communication
pathways, even if multiple providers are involved in the transmission.

\subsection{Production Order (Article 18)}
Each Party shall adopt such legislative and other measures as
may be necessary to empower its competent authorities to
order:
\begin{enumerate}
  \item a person in its territory to submit specified computer data in
    that person’s possession or control, which is stored in a computer
    system or a computer-data storage medium; and
  \item a service provider offering its services in the territory of the
    Party to submit subscriber information relating to such services in
    that service provider’s possession or control.
\end{enumerate}

\subsection{Subscriber Information (Article 18)}

For the purpose of this article, the term \textit{subscriber
information} means any information contained in the form of computer
data or any other form that is held by a service provider, relating to
subscribers of its services other than traffic or content data, and by
which can be established:

\begin{itemize}
    \item[(a)] the type of communication service used, the technical
      provisions taken thereto, and the period of service;
    \item[(b)] the subscriber's identity, postal or geographic
      address, telephone and other access number, billing and payment
      information, available on the basis of the service agreement or
      arrangement;
    \item[(c)] any other information on the site of the installation
      of communication equipment, available on the basis of the
      service agreement or arrangement.
\end{itemize}

\subsection{Search and Seizure of Stored Computer Data (Article 19)}

Each Party shall adopt such legislative and other measures as may be
necessary to empower its competent authorities to search or similarly
access:

\begin{itemize}
    \item[(a)] a computer system or part of it and computer data
      stored therein
    \item[(b)] a computer-data storage medium in which computer data
      may be stored in its territory.
\end{itemize}

\subsection{Real-time Collection of Traffic Data (Article 20)}

Authorities can collect or record traffic data in real time, either
directly or by requiring service providers to assist in the
collection.

\subsection{Interception of Content Data (Article 21)}
For serious offenses, authorities may intercept or record the content
of specific communications in real time, either directly or through
the cooperation of service providers.

\begin{boxH}
  From all those you only need to remember article 25 and 29.
\end{boxH}

\subsection{Mutual Assistance (Article 25)}
The Parties shall afford one another mutual assistance to the widest
extent possible for the purpose of investigations or proceedings
concerning criminal offences related to computer systems and data, or
for the collection of evidence in electronic form of a criminal
offence.

The Parties shall afford one another mutual assistance to the widest
extent possible for the purpose of investigations or proceedings
concerning criminal offences related to computer systems and data, or
for the collection of evidence in electronic form of a criminal
offence.

\subsection{Expedited preservation of stored computer data (Article
29)}
A Party may request another Party to order or otherwise obtain the
expeditious preservation of data stored by means of a computer system,
located within the territory of that other Party and in respect of
which the requesting Party intends to submit a request for mutual
assistance for the search or similar access, seizure or similar
securing, or disclosure of the data.

\subsection{Expedited disclosure of preserved traffic data (Article
30)}
Where, in the course of the execution of a request made pursuant to
Article 29 to preserve traffic data concerning a specific
communication, the requested Party discovers that a service provider
in another State was involved in the transmission of the
communication, the requested Party shall expeditiously disclose to the
requesting Party a sufficient amount of traffic data to identify that
service provider and the path through which the communication was
transmitted.

\subsection{Article 32 of the Cybercrime Convention (Budapest 2001)}
A Party may, without the authorisation of another Party:
\begin{itemize}
  \item[(a)] access publicly available (open source) stored computer data, regardless of where the data is located geographically, or
  \item[(b)] access or receive, through a computer system in its territory, stored computer data located in another Party, if the Party obtains the lawful and voluntary consent of the
    person who has the lawful authority to disclose the data to the Party through that computer system.
\end{itemize}

\subsection{Additional Legal Framework}

Regulation (EU) 2023/1543 of the European Parliament and of the 
Council of 12 July 2023 on European Production Orders and European 
Preservation Orders for electronic evidence in criminal proceedings 
and for the execution of custodial sentences following criminal 
proceedings.


Directive (EU) 2023/1544 of the European Parliament and of the 
Council of 12 July 2023 laying down harmonised rules on the 
designation of designated establishments and the appointment of 
legal representatives for the purpose of gathering electronic 
evidence in criminal proceedings.

In order to apply the rules in a consistent manner and to provide 
time for implementation and compliance, the Regulation applies 
from 18 August 2026. The Directive must be transposed into the 
national laws of the EU Member States by 18 February 2026.

