\chapter{Italian Law n. 48/2008}


The Budapest Convention on Cybercrime was issued by the Council of
Europe on November 23, 2001. 

\begin{boxH}
  Each country need to ratify a convenction (otherwise is only a
  general agreed principle, as in case of signature), therefore Italy
  ratified the it with Law n. 48 on March 18, 2008, which was
  published in the Official Gazette on April 4, 2008.
\end{boxH}

The ratification of the Budapest Convention by Italy through Law n.
48/2008 represents a critical step in modernizing the Italian legal
system to tackle cybercrime. This law aims to harmonize legal
standards across borders, emphasizing the importance of international
cooperation in addressing the rapidly evolving digital landscape.
However, challenges remain in adapting to rapid technological changes,
and continuous updates to legislative tools are necessary.

\section{Main Innovations Introduced by Italian Law n. 48/2008}

\begin{itemize}
    \item International harmonization of legislation to combat
      cybercrime.
    \item Reorganization of cybercrime offenses: amendments and
      integrations to the penal code, introducing new specific
      offenses.
    \item \textbf{Corporate Liability}: Extending liability under
      Legislative Decree 231/2001 to cover cybercrime offenses.
\end{itemize}

\section{Corporate Criminal Liability}

Italian Law n. 48/2008 extends corporate criminal liability to
offenses committed in the interest or for the benefit of the company,
as regulated by Legislative Decree 231/2001. 
\begin{boxH}
This applies to all legal entities, even foreign companies if the
offense is committed in Italy.
\end{boxH}

Manly, offences committed \textbf{in the interest} or \textbf{for the
benefit} of the company, and in most cases monetary fines,
interdictive sanctions  or publication of the sentence are applied to
the company.

The liability for the phisical persons lies to the person holding a
position of representation, management or direction or who exercise,
even if \textit{de facto}, management and control or a persons subject
to the control or monitoring activity of the Top Management.

The corporate liability, in Italy, is only applicable for some kinds
of crimes, such as tax offenses, money laundering, corruption, etc. In
case of corporate liaility, the company can demostrate to have a
compliance programm to ask for an exception, which is a set of
policies policies on the various topics, and demonstrate that all the
necessary measures have been taken to prevent the crime, and notheless
the crime has been committed.

\section{Further Innovations}

\begin{itemize}
    \item Establishment of a fund under the Ministry of the Interior
      to combat child pornography online.
    \item Updates to data retention laws, with reference to Directive
      2006/24/EC.
    \item Enhancement of international cooperation, especially
      concerning mutual assistance in cybercrime investigations.
    \item Acquisition of digital evidence: Updates to criminal
      procedure code for regulating the collection and use of digital
      evidence.
\end{itemize}

\section{Key Provisions and Their Implications for Digital Forensics}

\subsection{Major Procedural and Investigative Updates}

\subsubsection{International Cooperation}
The law enables the Ministry of the Interior or designated authorities to instruct internet and telecommunications providers to retain and protect traffic data for up to 90 days. This period can be extended to six months for specific investigative needs, but excludes the content of communications.

\subsubsection{Competence for Investigations and Prosecutions}
Cybercrime investigations and prosecutions are assigned to the Public Prosecutor’s Office at the Court of Appeal’s main district. This aims to improve coordination in cybercrime cases. However, initial issues arose due to the lack of transitional provisions for ongoing investigations, which were later addressed by Law n. 125 of July 2008.

\subsubsection{Service Providers' Role}
Service providers, including internet, telecommunications, and postal services, play a key role in combating cybercrime. Their responsibilities include:
\begin{itemize}
    \item Retention of traffic data and communication logs.
    \item Seizure of correspondence, including electronic communications, when linked to criminal investigations.
    \item Seizure of digital data, ensuring that original data is preserved while copies are made without modification.
\end{itemize}

\subsubsection{Legal Recognition of Computer Forensics}
The law introduces computer forensics as a significant element of investigative practices, with clear protocols for handling digital evidence. These practices will need continuous updates as technology evolves.

\subsection{Best Practices in Digital Evidence Handling}
The law advocates for "best practices" in digital evidence handling, emphasizing the importance of:
\begin{itemize}
    \item Acquiring evidence without altering the original device.
    \item Authenticating both the evidence and its digital copy.
    \item Ensuring that the examination process is repeatable.
    \item Maintaining impartiality in technical analysis.
\end{itemize}

\subsection{Email Seizures}
The law also includes provisions for seizing email communications, granting the same legal protections to both traditional and electronic mail.

\subsection{Changes to the Code of Criminal Procedure}
The law modifies the Code of Criminal Procedure to extend investigative measures such as inspections and seizures to digital environments. Noteworthy amendments include:
\begin{itemize}
    \item \textbf{Digital Inspections and Searches:} Technical measures must be implemented to preserve the integrity of the original data and avoid alterations.
    \item \textbf{Preservation Orders (Freezing):} These orders are introduced as a quick measure to secure digital evidence before it can be lost or tampered with.
\end{itemize}

\section{Law n. 48/2008: Areas for Improvement}

\subsection{Standardization of Digital Evidence Procedures}
Law n. 48/2008 established a unified approach for handling digital evidence in criminal proceedings. This standardization focuses on the acquisition, preservation, and presentation of digital evidence, ensuring its integrity and authenticity for admissibility in court.

\subsection{Judicial Expertise and Training in Digital Forensics}
The implementation of this law has increased the responsibility of legal professionals to understand digital forensics. There is a growing need for specialized training in this area to avoid misinterpretation of digital evidence.

\subsection{Legal Certainty and Data Integrity}
Ensuring the integrity and authenticity of digital data is crucial. Law n. 48/2008 improved the reliability of digital evidence in court by requiring that data remain unaltered during acquisition and preservation. However, further refinement of these procedures is necessary, especially with the rise of more sophisticated cyber threats.

\section{Case Studies and Practical Applications of Italian Law n. 48/2008}

\subsection{Case Study 1: Data Seizure and Service Providers}
Phishing cases often involve the seizure of data from service providers to
trace illegal transactions. Although no specific case is named, this method is
frequently used in investigations of fraud and financial crimes.

\subsection{Case Study 2: Organized Crime and Communication Monitoring}
A significant case involved intercepting communications of mafia organizations
in Italy. This was part of a broader initiative coordinated with Europol and
Interpol, utilizing advanced digital forensic techniques to collect evidence
from encrypted messages. In osme case, getting cooperation with cross-contry 
orgationations its easier than with the local ones. For example
interpol has a contact inside the TOR organization, which helps a lot
in some instances.

\subsection{Case Study 3: Cyberstalking}
In several cyberstalking cases in Italy, digital forensics were employed to
trace online threats' origins. Investigators successfully identified offenders
through data traffic analysis and IP identification, making this a common
approach in Italian jurisprudence.

\section{Types of Corporate Investigations}
There are different kind of corporate investigations, such as:
\begin{itemize}
    \item \textbf{Unfair Competition:} Investigating unethical practices by
      rival businesses.
    \item \textbf{Industrial Espionage:} Uncovering the theft of trade secrets
      and proprietary information.
    \item \textbf{Employee Misconduct:} Addressing violations of company
      policies and contracts.
    \item \textbf{Intellectual Property Infringement:} Protecting copyrights,
      trademarks, and patents from unauthorized use.
\end{itemize}

\section{Man-in-the-Middle (MITM) Attacks}
\begin{boxH}
  This is the only case that will be asked at the exam.
\end{boxH}
MITM attacks represent a silent and sophisticated form of cybercrime where an
attacker intercepts communication between two parties. This allows criminals to
monitor, read, and modify messages undetected, often targeting businesses
involved in international trade.

\subsection{The Mechanics of MITM Attacks}
\begin{itemize}
    \item \textbf{Initial Breach:} Hackers compromise a company's email system
      through methods such as phishing, brute force attacks, or trojans.
    \item \textbf{Monitoring Phase:} Attackers observe communications over
      time, gathering information on business practices and relationships.
    \item \textbf{Interception:} At an opportune moment, criminals intercept
      and alter payment instructions, redirecting funds to their accounts.
    \item \textbf{Execution:} Unaware of the fraud, victims transfer funds to
      the fraudulent account, often resulting in significant financial loss.
\end{itemize}

\subsection{Legal Implications of MITM Attacks}
\textbf{Criminal Perspective:} MITM attacks can be classified as fraud under
Article 640 of the Italian Criminal Code, involving deception through false
emails and documents. \textbf{Identity Theft:} Such attacks may also fall under
identity substitution (Article 494) and computer fraud (Article 640 ter),
particularly with recent legislative changes.

\subsubsection{Jurisdictional Challenges}
Prosecution of MITM attacks is often complicated by jurisdictional issues and
time constraints, as attackers frequently operate from countries with limited
judicial cooperation.

\subsection{Civil Recourse and Bank Responsibilities}
\paragraph{Immediate Action} Victims should promptly request payment reversals
while funds remain in the destination account. Some European banks may assist
by freezing accounts and returning funds.

\paragraph{Bank Liability} European regulations, such as the Payment Service
Regulation and Italian Legislative Decree 27 January 2010, n. 11, often shield
banks from liability, even when account holder details do not match.

\paragraph{Regulatory Gaps} Current regulations may be insufficient in an era
of fast online payments, where financial intermediaries predominantly control
transaction verification.

