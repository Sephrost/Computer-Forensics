\section{Forensics scenarios}
When doing a forensic investigation, there can be different scenarios
that can be encountered. Some common scenarios are:
\begin{itemize}
  \item internet abuse from employee
  \item computer-aided frauds
  \item data unauthorized manipulation, like data theft or disclosure
  \item computer/network damage assessment
  \item …and any time digital evidences may be involved in an incident
\end{itemize}
\section{Investigation phases}
The investigation process can be divided into several phases, at least
from a technical point of view. Those depends on the standards that
are followed in a given country where the investigation is taking 
place(fore example in America the NIST standards are followed).
\subsection{Identification}
Its the first step of the investigation, an it take place when the
crime scene has been accessed. The goal is to identify which are the
potential sources of relevant that, which will be used as digital
evidence if relevant.\\
Most of the time we have an overwhelming amount of data, but most of
it is not relevant to the investigation. Reducing the amount of data 
is also a goal of this phase. On the other hand, it is also possible
to miss some important data, so it is important to be careful.\\
It is important to recognize (all relevant) data sources before any
acquisition begins, like:
\begin{itemize}
  \item hard drives (HDD/SSD)
  \item memory (RAM)
  \item mobile devices (smartphones, tablets)
  \item cloud storage
  \item network traffic
  \item removable media (USB drives, DVDs)
  \item \dots 
\end{itemize}
The steps to follow in this phase are:
\begin{itemize}
  \item if possible, acquire data before reaching the crime
    scene(pre-analysis), instructing the staff to identify possible
    sources of evidence
  \item perform an \textbf{initial survey} of the scene (physical or
    network environment)
  \item \textbf{identify key devices} and data locations (local
    storage, remote servers, cloud services)
  \item \textbf{check} for \textbf{connected devices}, including
    peripherals like printers, removable media, or network-attached
    devices
  \item \textbf{map} all potential \textbf{data sources} using network
    topology
\end{itemize}
Pay much attention to the ephemeral storage of data, with reference to
the order of volatility. The order of volatility is a concept that
refers to the order in which data should be collected, based on how 
long it will be available.
\subsection{Collection}
After the identification phase, one must physically or remotely taking
possession of the evidence (e.g. a computer) and its connection (e.g.
Network or physical, like USB disk). The timing is most crucial in
this phase, while also maintaining the integrity of the evidence, to
minimize the risk of evidence tampering or data loss.\\
The steps to follow in this phase are:
\begin{itemize}
  \item isolate devices to prevent them from being tampered with
    remotely (e.g., disconnect them from the network)
  \item use devices to block external communication for mobile or
    wireless devices ( e.g. faraday bags)
  \item use network isolation tools for virtual and cloud environments
    to prevent remote access ( e.g security groups, virtual private
    cloud, firewall rule)
\end{itemize}
Its important to ensure the integrity of the evidence while
maintaining the system running, because shutting it down can cause the
loss of volatile data(e.g. RAM).
\subsection{Acquisition}
electronically retrieving data by running various 
CF tools and software suite. Its a separate phase from the collection 
phase.
\subsection{Evaluation}
Now that all the data has been collected, it is time to evaluate it.
In this phase the data is analyzed to substantiate claims and to
determine how they could be used against the suspect.
\subsection{Presentation}
At last, the evidence is presented in a clear and understandable way
to the court, in a manner which is suitable for lawyers, non-technical 
staff and the law.
