\chapter{Introduction to digital forensics}
First of all, we need to understand some basic concepts. \\
\textbf{Forensics analysis} is the process of investigating and
analyzing information to gather evidence to solve legal problems. In
order to do that, the collections, presevation and analysis or
presentation of digital evidence is required to support the
investigation. \textbf{Computer forensics} does just that.\\
Forensic science is not a new field, it has been around since the
ancient times. The first recorded use of forensic science was around
1900 BC in Babylon, where fingerprints were used to identify the 
author of a clay tablet. Digital forensics is a relatively new field,
it started in the 1980s with the advent of personal computers, for
example the first convicted person for digital crimes was Robert
Tappan Morris, who created the Morris Worm in 1988, and found out by
analyzing computer logs and network activity.
\begin{boxH}
  \textbf{Computer Forensics} is the \textbf{discipline} that combines elements
  of law and computer science to collect and analyze data from computer
  systems, networks, wireless communications, and storage devices in a way that
  is admissible as evidence in a court of law.
\end{boxH}
